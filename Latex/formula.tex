\documentclass{article}
\usepackage{mathtools}
\usepackage{amssymb}
\usepackage{amsthm}
\usepackage{mathrsfs}

\DeclareMathOperator{\ce}{ce}
\DeclareMathOperator{\se}{se}
\DeclareMathOperator{\Ce}{Ce}
\DeclareMathOperator{\Se}{Se}

\begin{document}

Differential Equation:
\begin{equation}
    \frac{d^2y}{dx^2} + (a - 2q\cos(2x))y = 0
\end{equation}

For all $q$ there are special values $a_n(q)$ and $b_n(q)$ such that the solutions are periodic with period $\pi$ or $2\pi$.

If $q=0$ this equation then reduces to $y'' + ay = 0$ which has solutions $\cos(\sqrt{a}x)$ and $\sin(\sqrt{a}x)$. These solutions have period $\pi$ or $2\pi$ if $a = n^2$. In this case we say $a(0) = a_n(0) = n^2 = b_n(0)=b(0)$. 

In general $a_n(q) \neq b_n(q)$.

We also have:

\begin{equation}
    a_0(q) < b_1(q) < a_1(q) < b_2(q) < \dots
\end{equation}

When $q \neq 0$ we denote the periodic solutions to Mathieu's Equation
\begin{center}
    $\ce_n(x, q)$, $\se_n(x, q)$   
\end{center}

\begin{center}
    \begin{tabular}{|c|c|c|}
    \hline
    Function & Period & Symmetry \\ \hline
    $\ce_{2r}$ & $\pi$ & even \\ \hline
    $\ce_{2r+1}$ & $2\pi$ & even \\ \hline
    $\se_{2r}$ & $\pi$ & odd \\ \hline
    $\se_{2r+1}$ & $2\pi$ & odd \\ \hline
    \end{tabular}
\end{center}

Fourier series

Given the previous properties we may expand the Mathieu functions as fourier series containing only either $\cos$ or $\sin$ terms and either only even or odd multiples.

\begin{equation}
    \ce_{2r}(x, q) = \sum_{s=0}^{\infty}{A_{2s}^{2r}(q)\cos(2sx)}
\end{equation}

\begin{equation}
    \ce_{2r+1}(x, q) = \sum_{s=0}^{\infty}{A_{2s+1}^{2r+1}(q)\cos((2s+1)x)}
\end{equation}

\begin{equation}
    \se_{2r}(x, q) = \sum_{s=1}^{\infty}{B_{2s}^{2r}(q)\sin(2sx)}
\end{equation}

\begin{equation}
    \se_{2r+1}(x, q) = \sum_{s=0}^{\infty}{B_{2s+1}^{2r+1}(q)\sin((2s+1)x)}
\end{equation}

Orthogonality

As the Mathieu equation is of Sturm-Liouville form, for fixed $q$ the solutions $\ce_n$ and $\se_n$ must be orthogonal over $[0, 2\pi]$:

\begin{equation}
    \int_0^{2\pi}\ce_m(x, q)\ce_n(x, q)dx = \pi \delta_{mn}
\end{equation}

\begin{equation}
    \int_0^{2\pi}\se_m(x, q)\se_n(x, q)dx = \pi \delta_{mn}
\end{equation}

\begin{equation}
    \int_0^{2\pi}\ce_m(x, q)\se_n(x, q)dx = 0
\end{equation}

\begin{proof}
    Suppose we have two solutions to Mathieu's equation $y_1(x, q)$ and $y_2(x, q)$ with corresopnding eigenvalues $\lambda_1$, $\lambda_2$ where $\lambda_1 \neq \lambda_2$, then we have:
    \begin{equation}
        \frac{d^2 y_1}{dx^2} + (\lambda_1 - 2 q \cos(x)) y_1 = 0
    \end{equation}

    \begin{equation}
        \frac{d^2 y_2}{dx^2} + (\lambda_2 - 2 q \cos(x)) y_2 = 0
    \end{equation}

    Multiply the first by $y_2$ and the second by $y_1$ and then subratct:
    \begin{equation}
        y_2 \frac{d^2 y_1}{dx^2} - y_1 \frac{d^2 y_2}{dx^2} = 
        (\lambda_1 - \lambda_2) y_1 y_2
    \end{equation}
    Integrate from $0$ to $2\pi$:
    \begin{equation}
        \int_{0}^{2\pi} \left( y_2 \frac{d^2y_1}{dx^2} - y_1 \frac{d^2y_2}{dx^2} \right) dx = \int_{0}^{2\pi} (\lambda_1 - \lambda_2) y_1 y_2 dx
    \end{equation}
    We have:
    \begin{align}
       y_2 \frac{d^2y_1}{dx^2} - y_1 \frac{d^2y_2}{dx^2} & = 
       y_2 \frac{d^2y_1}{dx^2} + \frac{dy_1}{dx} \frac{dy_2}{dx} - \frac{dy_1}{dx} 
       \frac{dy_2}{dx} - y_1 \frac{d^2y_2}{dx^2} \\
        & = \frac{d}{dx} \left( y_1 \frac{dy_2}{dx} - y_2 \frac{dy_1}{dx} \right)
    \end{align}
    Hence:
    \begin{equation}
        \left[ y_1 \frac{dy_2}{dx} - y_2 \frac{dy_1}{dx} \right]_{0}^{2\pi} = 
        (\lambda_1 - \lambda_2) \int_{0}^{2\pi} y_1 y_2 dx
    \end{equation}
    However, we know that $y_1$ and $y_2$ (and hence $\frac{dy_1}{dx}$, $\frac{dy_1}{dx}$) are periodic with period $2\pi$ (or $\pi$ makes no difference) so the LHS is identically zero:
    \begin{equation}
        0 = (\lambda_1 - \lambda_2) \int_{0}^{2\pi} y_1 y_2 dx
    \end{equation}
    As $\lambda_1 - \lambda_2 \neq 0$ we must have:
    \begin{equation}
        \int_{0}^{2\pi} y_1 y_2 dx = 0
    \end{equation}
    If $\lambda_1 = \lambda_2$ then $y_1 = y_2$ so 
    \begin{equation}
        \int_{0}^{2\pi} y_1^2dx \geq 0
    \end{equation}
    Which gives us the freedom to choose this value equal to $\pi$.
\end{proof}

Using the orthogonality of the funuctions we can find a suitable normalisation for the fouurier coefficients.

Using the $\sin$ and $\cos$ orthogonality rules:

\begin{equation}
    \int_0^{2\pi} \cos(mx) \cos(nx) dx = 
    \begin{cases}
        2\pi, & \text{if}\ m = n = 0 \\
        \pi \delta_{mn}, & \text{otherwise}
    \end{cases}
\end{equation}
\begin{equation}
    \int_0^{2\pi} \sin(mx) \sin(nx) dx = 
    \pi \delta_{mn}
\end{equation}
 We can derive the following:

\begin{equation}
    2\left(A_0^{2r}\right)^2 + \sum_{s=1}^\infty \left(A_{2s}^{2r}\right)^2 = 1
\end{equation}

\begin{equation}
    \sum_{s=0}^\infty \left(A_{2s+1}^{2r+1}\right)^2 = 1
\end{equation}

\begin{equation}
    \sum_{s=1}^\infty \left(B_{2s}^{2r}\right)^2 = 1
\end{equation}

\begin{equation}
    \sum_{s=0}^\infty \left(B_{2s+1}^{2r+1}\right)^2 = 1
\end{equation}

Matrix Formulation

From the differential equation and using the fourier expansions of the Mathieu functions we can derive an infinite matrix formulation of the problem. From this one can calculate the eigenvalue $a_n$, $b_n$, as well as the coefficients.

$\ce_{2r}(x, q)$

\begin{align}
    0 & = \frac{d^2\ce_{2r}}{dx^2} + (a_{2n} - 2q\cos(2x))\ce_{2n} \\
    & = \frac{d^2}{dx^2}\left(\sum_{s=0}^{\infty}{A_{2s}^{2r}(q)\cos(2sx)}\right) + (a_{2n} - 2q\cos(2x))\sum_{s=0}^{\infty}{A_{2s}^{2r}(q)\cos(2sx)} \\
    & = -\sum_{s=0}^{\infty}{A_{2s}^{2r}(q) (2s)^2 \cos(2sx)} + a_{2n} \sum_{s=0}^{\infty}{A_{2s}^{2r}(q)\cos(2sx)} - 2q\cos(2x)\sum_{s=0}^{\infty}{A_{2s}^{2r}(q)\cos(2sx)}
\end{align}

Using 

\begin{equation}
    \cos(2x) \cos(2nx) = \frac{1}{2} \left[\cos(2(n+1)x) + \cos(2(n-1)x)\right]
\end{equation}

We can equate the coefficients to get:

\begin{align}
    a A_0 - q A_2= & 0 \\
    -2qA_0 + (a - 4) A_2 - q A_4 = & 0 \\
    -qA_{2k-2} + (a - (2k)^2) A_{2k} - q A_{2k+2} = & 0
\end{align}

Similarly for the other functions: \\
$\ce_{2r+1}$

\begin{align}
    (a - 1 - q) A_1 - q A_3 = & 0 \\
    -qA_{2k-1} + (a - (2k+1)^2) A_{2k+1} - q A_{2k+3} = & 0
\end{align}

$\se_{2r}$

\begin{align}
    (b - 4) B_2 - q B_4 = & 0 \\
    -qB_{2k-2} + (b - (2k)^2) B_{2k} - q B_{2k+2} = & 0
\end{align}

$\se_{2r+1}$

\begin{align}
    (b - 1 + q) B_1 - q B_3 = & 0 \\
    -qB_{2k-1} + (b - (2k+1)^2) B_{2k+1} - q B_{2k+3} = & 0
\end{align}

\begin{equation}
    \left(
    \begin{matrix}
        0 & \sqrt{2}q & 0 & 0 & \dots \\
        \sqrt{2}q & 4 & q & 0 & \dots \\
        0 & q & 16 & q & \dots \\
        0 & 0 & q & 36 & \dots \\
        \vdots & \vdots & \vdots & \vdots & \ddots \\

    \end{matrix}
    \right)
    \left(
    \begin{matrix}
        \sqrt{2}A_0^{2r} \\
        A_2^{2r}\\
        A_4^{2r}\\
        A_6^{2r}\\
        \vdots
    \end{matrix}
    \right)
    = a_{2r}(q)
    \left(
    \begin{matrix}
        \sqrt{2}A_0^{2r} \\
        A_2^{2r}\\
        A_4^{2r}\\
        A_6^{2r}\\
        \vdots
    \end{matrix}
    \right)
\end{equation}

\begin{equation}
    \left(
    \begin{matrix}
        1+q & q & 0 & 0 & \dots \\
        q & 9 & q & 0 & \dots \\
        0 & q & 25 & q & \dots \\
        0 & 0 & q & 49 & \dots \\
        \vdots & \vdots & \vdots & \vdots & \ddots \\

    \end{matrix}
    \right)
    \left(
    \begin{matrix}
        A_1^{2r+1} \\
        A_3^{2r+1}\\
        A_5^{2r+1}\\
        A_7^{2r+1}\\
        \vdots
    \end{matrix}
    \right)
    = a_{2r+1}(q)
    \left(
    \begin{matrix}
        A_1^{2r+1} \\
        A_3^{2r+1}\\
        A_5^{2r+1}\\
        A_7^{2r+1}\\
        \vdots
    \end{matrix}
    \right)
\end{equation}

\begin{equation}
    \left(
    \begin{matrix}
        4 & q & 0 & 0 & \dots \\
        q & 16 & q & 0 & \dots \\
        0 & q & 36 & q & \dots \\
        0 & 0 & q & 64 & \dots \\
        \vdots & \vdots & \vdots & \vdots & \ddots \\

    \end{matrix}
    \right)
    \left(
    \begin{matrix}
        B_2^{2r} \\
        B_4^{2r}\\
        B_6^{2r}\\
        B_8^{2r}\\
        \vdots
    \end{matrix}
    \right)
    = b_{2r}(q)
    \left(
        \begin{matrix}
            B_2^{2r} \\
            B_4^{2r}\\
            B_6^{2r}\\
            B_8^{2r}\\
            \vdots
        \end{matrix}
    \right)
\end{equation}

\begin{equation}
    \left(
    \begin{matrix}
        1-q & q & 0 & 0 & \dots \\
        q & 9 & q & 0 & \dots \\
        0 & q & 25 & q & \dots \\
        0 & 0 & q & 49 & \dots \\
        \vdots & \vdots & \vdots & \vdots & \ddots \\

    \end{matrix}
    \right)
    \left(
    \begin{matrix}
        B_1^{2r+1} \\
        B_3^{2r+1}\\
        B_5^{2r+1}\\
        B_7^{2r+1}\\
        \vdots
    \end{matrix}
    \right)
    = b_{2r+1}(q)
    \left(
        \begin{matrix}
            B_1^{2r+1} \\
            B_3^{2r+1}\\
            B_5^{2r+1}\\
            B_7^{2r+1}\\
            \vdots
        \end{matrix}
    \right)
\end{equation}

https://www.biblioteca.org.ar/libros/90028.pdf

We can also expand $\cos(x)$ and $\sin(x)$ in terms of $\ce_n(x, q)$ and $\se_n(x, q)$:

\begin{equation}
    \cos(2sx) = \sum_{r=0}^\infty \overline{A}_{2r}^{2s}(q)\ce_{2r}(x,q)  
\end{equation}

\begin{equation}
    \cos((2s+1)x) = \sum_{r=0}^\infty \overline{A}_{2r+1}^{2s+1}(q)\ce_{2r+1}(x,q)  
\end{equation}

\begin{equation}
    \sin(2sx) = \sum_{r=0}^\infty \overline{B}_{2r}^{2s}(q)\se_{2r}(x,q)  
\end{equation}

\begin{equation}
    \sin((2s+1)x) = \sum_{r=0}^\infty \overline{B}_{2r+1}^{2s+1}(q)\se_{2r+1}(x,q)  
\end{equation}

By fouriers stuff we find:

\begin{equation}
    A_{2s}^{2r}(q) = \overline{A}_{2r}^{2s}(q)
\end{equation}

Except:
\begin{equation}
    2A_{0}^{2r}(q) = \overline{A}_{2r}^{0}(q)
\end{equation}

\begin{equation}
    A_{2s+1}^{2r+1}(q) = \overline{A}_{2r+1}^{2s+1}(q)
\end{equation}

\begin{equation}
    B_{2s}^{2r}(q) = \overline{B}_{2r}^{2s}(q)
\end{equation}

\begin{equation}
    B_{2s+1}^{2r+1}(q) = \overline{B}_{2r+1}^{2s+1}(q)
\end{equation}

Differential Equation modified:
\begin{equation}
    \frac{d^2y}{dx^2} - (a - 2q\cosh(2x))y = 0
\end{equation}

This equation relates to that of the angular mathieu functions via $x \rightarrow ix$ and hence the corresponding solutions are:

\begin{equation}
    \ce_n(ix, q)
\end{equation}
\begin{equation}
    -i \se_n(ix, q)
\end{equation}

Provided that $a = a_n(q)$ or $a = b_n(q)$.

We can write these solutions as:

\begin{equation}
    \Ce_n(x, q) = \ce_n(ix, q)
\end{equation}

\begin{equation}
    \Se_n(x, q) = -i \se_n(ix, q)
\end{equation}

Using these we can rewrite the fourier expansions as:

\begin{equation}
    \Ce_{2r}(x, q) = \sum_{s=0}^{\infty}{A_{2s}^{2r}(q)\cosh(2sx)}
\end{equation}

\begin{equation}
    \Ce_{2r+1}(x, q) = \sum_{s=0}^{\infty}{A_{2s+1}^{2r+1}(q)\cosh((2s+1)x)}
\end{equation}

\begin{equation}
    \Se_{2r}(x, q) = \sum_{s=1}^{\infty}{B_{2s}^{2r}(q)\sinh(2sx)}
\end{equation}

\begin{equation}
    \Se_{2r+1}(x, q) = \sum_{s=0}^{\infty}{B_{2s+1}^{2r+1}(q)\sinh((2s+1)x)}
\end{equation}

However this isn't the best formulation of these functions as $\cosh$ and $\sinh$ grom very rapidly. Instead we should expand these in terms of bessel functions.

We know that the bessel functions are orthonormal in two ways:

\begin{equation}
    \int_0^1 J_n(\alpha_{ni}x) J_n(\alpha_{nj}x) dx = \delta_{ij} \frac{J^{\prime}_n(\alpha_{ni})}{2}
\end{equation}

\begin{align}
    \int_0^\infty J_n(x) J_m(x) \frac{dx}{x} & = \delta_{nm} \frac{1}{2n} \\
    \int_{-\infty}^\infty J_n(e^u) J_m(e^u) du & = \delta_{nm} \frac{1}{2n} \\
    \int_{0}^\infty J_n(2 \sqrt{q} \sinh(v)) J_m(2 \sqrt{q} \sinh(v)) \coth(v) dv & = \delta_{nm} \frac{1}{2n}
\end{align}

GETS FUNNY AT $m = n = 0$

Using the second one as a basis for $L^2[0,\infty]$ we can expand the modified mathieu functions as a sum of Bessel functions:

Let y 
\begin{equation}
    \Ce_{n}(x) = \sum_{0}^{\infty} c_n J_{n} \left(2 \sqrt{q} \sinh(x) \right)
\end{equation}



\end{document}
