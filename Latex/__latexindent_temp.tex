\documentclass{article}
\usepackage{mathtools}
\usepackage{amssymb}
\usepackage{amsthm}
\usepackage{mathrsfs}

\begin{document}

Differential Equation:
\begin{equation}
    \frac{d^2y}{dx^2} + (a - 2q\cos(x))y = 0
\end{equation}

For all $q$ there are special values $a_n(q)$ and $b_n(q)$ such that the solutions are periodic with period $\pi$ or $2\pi$.

If $q=0$ this equation then reduces to $y'' + ay = 0$ which has solutions $\cos(\sqrt{a}x)$ and $\sin(\sqrt{a}x)$. These solutions have period $\pi$ or $2\pi$ if $a = n^2$. In this case we say $a(0) = a_n(0) = n^2 = b_n(0)=b(0)$. 

In general $a_n(q) \neq b_n(q)$.

We also have:

\begin{equation}
    a_0(q) < b_1(q) < a_1(q) < b_2(q) < \dots
\end{equation}

When $q \neq 0$ we denote the periodic solutions to Mathieu's Equation
\begin{center}
    $ce_n(x, q)$, $se_n(x, q)$   
\end{center}

We may expand these as fourier series:

\begin{equation}
    ce_{2r}(x, q) = \sum_{s=0}^{\infty}{A_{2s}^{2r}(q)\cos(2sx)}
\end{equation}

\begin{equation}
    ce_{2r+1}(x, q) = \sum_{s=0}^{\infty}{A_{2s+1}^{2r+1}(q)\cos((2s+1)x)}
\end{equation}

\begin{equation}
    se_{2r}(x, q) = \sum_{s=0}^{\infty}{B_{2s}^{2r}(q)\sin(2sx)}
\end{equation}

\begin{equation}
    se_{2r+1}(x, q) = \sum_{s=0}^{\infty}{B_{2s+1}^{2r+1}(q)\sin((2s+1)x)}
\end{equation}

\begin{center}
    \begin{tabular}{|c|c|c|}
    \hline
    Function & Period & Symmetry \\ \hline
    $ce_{2r}$ & $\pi$ & even \\ \hline
    $ce_{2r+1}$ & $2\pi$ & even \\ \hline
    $se_{2r}$ & $\pi$ & odd \\ \hline
    $se_{2r+!}$ & $2\pi$ & odd \\ \hline
    \end{tabular}
\end{center}

Orthogonality

As the Mathieu equation is of Sturm-Liouvill form, for fixed q the solutions $ce_n$ and $se_n$ must be orthogonal over $[0, 2\pi]$:

\begin{equation}
    \int_0^{2\pi}ce_m(x, q)ce_n(x, q)dx = \pi \delta_{mn}
\end{equation}

\begin{equation}
    \int_0^{2\pi}se_m(x, q)se_n(x, q)dx = \pi \delta_{mn}
\end{equation}

\begin{equation}
    \int_0^{2\pi}ce_m(x, q)se_n(x, q)dx = 0
\end{equation}

\begin{proof}
    Suppose we have two solutions to Mathieu's equation $y_1(x, q)$ and $y_2(x, q)$ with corresopnding eigenvalues $\lambda_1$, $\lambda_2$ where $\lambda_1 \neq \lambda_2$, then we have:
    \begin{equation}
        \frac{d^2 y_1}{dx^2} + (\lambda_1 - 2 q \cos(x)) y_1 = 0
    \end{equation}

    \begin{equation}
        \frac{d^2 y_2}{dx^2} + (\lambda_2 - 2 q \cos(x)) y_2 = 0
    \end{equation}

    Multiply the first by $y_2$ and the second by $y_1$ and then subratct:

    \begin{equation}
        y_2 \frac{d^2 y_1}{dx^2} - y_1 \frac{d^2 y_2}{dx^2} = (\lambda_1 - \lambda_2) y_1 y_2
    \end{equation}

    Integrate from $0$ to $2\pi$:

    \begin{equation}
        \int_{0}^{2\pi} \left( y_2 \frac{d^2y_1}{dx^2} - y_1 \frac{d^2y_2}{dx^2} \right) dx = \int_{0}^{2\pi} (\lambda_1 - \lambda_2) y_1 y_2 dx
    \end{equation}

    We have:

        \begin{align}
            y_2 \frac{d^2y_1}{dx^2} - y_1 \frac{d^2y_2}{dx^2} & = y_2 \frac{d^2y_1}{dx^2} + \frac{dy_1}{dx} \frac{dy_2}{dx} - \frac{dy_1}{dx} \frac{dy_2}{dx} - y_1 \frac{d^2y_2}{dx^2} 
        \end{align}

\end{proof}

\begin{equation}
    \left(A_0^{2r}\right)^2 + \sum_{s=0}^\infty \left(A_{2s}^{2r}\right)^2 = 1
\end{equation}

\begin{equation}
    \sum_{s=0}^\infty \left(A_{2s+1}^{2r+1}\right)^2 = 1
\end{equation}

\begin{equation}
    \sum_{s=0}^\infty \left(B_{2s}^{2r}\right)^2 = 1
\end{equation}

\begin{equation}
    \sum_{s=0}^\infty \left(B_{2s+1}^{2r+1}\right)^2 = 1
\end{equation}

\begin{equation}
    \left(
    \begin{matrix}
        0 & \sqrt{2}q & 0 & 0 & \dots \\
        \sqrt{2}q & 4 & q & 0 & \dots \\
        0 & q & 16 & q & \dots \\
        0 & 0 & q & 36 & \dots \\
        \vdots & \vdots & \vdots & \vdots & \ddots \\

    \end{matrix}
    \right)
    \left(
    \begin{matrix}
        \sqrt{2}A_0^{2r} \\
        A_2^{2r}\\
        A_4^{2r}\\
        A_6^{2r}\\
        \vdots
    \end{matrix}
    \right)
    = a_{2r}(q)
    \left(
    \begin{matrix}
        \sqrt{2}A_0^{2r} \\
        A_2^{2r}\\
        A_4^{2r}\\
        A_6^{2r}\\
        \vdots
    \end{matrix}
    \right)
\end{equation}

\begin{equation}
    \left(
    \begin{matrix}
        1+q & q & 0 & 0 & \dots \\
        q & 9 & q & 0 & \dots \\
        0 & q & 25 & q & \dots \\
        0 & 0 & q & 49 & \dots \\
        \vdots & \vdots & \vdots & \vdots & \ddots \\

    \end{matrix}
    \right)
    \left(
    \begin{matrix}
        A_1^{2r+1} \\
        A_3^{2r+1}\\
        A_5^{2r+1}\\
        A_7^{2r+1}\\
        \vdots
    \end{matrix}
    \right)
    = a_{2r+1}(q)
    \left(
    \begin{matrix}
        A_1^{2r+1} \\
        A_3^{2r+1}\\
        A_5^{2r+1}\\
        A_7^{2r+1}\\
        \vdots
    \end{matrix}
    \right)
\end{equation}

\begin{equation}
    \left(
    \begin{matrix}
        4 & q & 0 & 0 & \dots \\
        q & 16 & q & 0 & \dots \\
        0 & q & 36 & q & \dots \\
        0 & 0 & q & 64 & \dots \\
        \vdots & \vdots & \vdots & \vdots & \ddots \\

    \end{matrix}
    \right)
    \left(
    \begin{matrix}
        B_2^{2r} \\
        B_4^{2r}\\
        B_6^{2r}\\
        B_8^{2r}\\
        \vdots
    \end{matrix}
    \right)
    = b_{2r}(q)
    \left(
        \begin{matrix}
            B_2^{2r} \\
            B_4^{2r}\\
            B_6^{2r}\\
            B_8^{2r}\\
            \vdots
        \end{matrix}
    \right)
\end{equation}

\begin{equation}
    \left(
    \begin{matrix}
        1-q & q & 0 & 0 & \dots \\
        q & 9 & q & 0 & \dots \\
        0 & q & 25 & q & \dots \\
        0 & 0 & q & 49 & \dots \\
        \vdots & \vdots & \vdots & \vdots & \ddots \\

    \end{matrix}
    \right)
    \left(
    \begin{matrix}
        B_1^{2r+1} \\
        B_3^{2r+1}\\
        B_5^{2r+1}\\
        B_7^{2r+1}\\
        \vdots
    \end{matrix}
    \right)
    = b_{2r+1}(q)
    \left(
        \begin{matrix}
            B_1^{2r+1} \\
            B_3^{2r+1}\\
            B_5^{2r+1}\\
            B_7^{2r+1}\\
            \vdots
        \end{matrix}
    \right)
\end{equation}

https://www.biblioteca.org.ar/libros/90028.pdf

We can also expand $\cos(x)$ and $\sin(x)$ in terms of $ce_n(x, q)$ and $se_n(x, q)$:

\begin{equation}
    \cos(2sx) = \sum_{r=0}^\infty \overline{A}_{2r}^{2s}(q)ce_{2r}(x,q)  
\end{equation}

\begin{equation}
    \cos((2s+1)x) = \sum_{r=0}^\infty \overline{A}_{2r+1}^{2s+1}(q)ce_{2r+1}(x,q)  
\end{equation}

\begin{equation}
    \sin(2sx) = \sum_{r=0}^\infty \overline{B}_{2r}^{2s}(q)se_{2r}(x,q)  
\end{equation}

\begin{equation}
    \sin((2s+1)x) = \sum_{r=0}^\infty \overline{B}_{2r+1}^{2s+1}(q)se_{2r+1}(x,q)  
\end{equation}

By fouriers stuff we find:

\begin{equation}
    A_{2s}^{2r}(q) = \overline{A}_{2r}^{2s}(q)
\end{equation}

Except:
\begin{equation}
    2A_{0}^{2r}(q) = \overline{A}_{2r}^{0}(q)
\end{equation}

\begin{equation}
    A_{2s+1}^{2r+1}(q) = \overline{A}_{2r+1}^{2s+1}(q)
\end{equation}

\begin{equation}
    B_{2s}^{2r}(q) = \overline{B}_{2r}^{2s}(q)
\end{equation}

\begin{equation}
    B_{2s+1}^{2r+1}(q) = \overline{B}_{2r+1}^{2s+1}(q)
\end{equation}


\end{document}
